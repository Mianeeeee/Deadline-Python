\documentclass[12pt]{article}
\usepackage[margin=1in]{geometry}
\usepackage{graphicx}
\usepackage{amsmath}
\usepackage{hyperref}
\usepackage{titlesec}
\titleformat{\section}{\normalfont\Large\bfseries}{\thesection.}{1em}{}
\titleformat{\subsection}{\normalfont\large\bfseries}{\thesubsection.}{1em}{}
\titleformat{\subsubsection}{\normalfont\normalsize\bfseries}{\thesubsubsection.}{1em}{}

\title{\textbf{Report Justification and Results}}
\author{Vu Viet Thai – B23DCCE085}
\date{}

\begin{document}
\maketitle

\section{Problem 1:}

\textbf{Approach:} \textit{problem-1} has the task of collecting data about players who play in the \textit{Premier League} on the website \href{https://fbref.com/en/}{FBref.com}. It is designed to:

\begin{itemize}
    \item \textit{Access the website:} Connect to FBref to obtain data.
    \item \textit{Search for the league:} Identify the Premier League page.
    \item \textit{Get team data:} Browse the list of teams, then access each team page.
    \item \textit{Extract player data:} Collect various types of statistics, including: 
    
    \begin{itemize}
        \item Standard Stats.
        \item Goalkeeping.
        \item Shooting.
        \item Passing.
        \item Goal and Shot Creation.
        \item Defensive Actions.
        \item Possession.
        \item Miscellaneous Stats.
    \end{itemize}
    
    \item \textit{Save results:} The final data are saved to a CSV file (results.csv).
\end{itemize}

\textbf{Results:} After running the program, a data table containing detailed information about the \textit{Premier League} players will be saved in the results.csv file, including:

\begin{itemize}
    \item \textbf{Standard Stats:} Player name, Nation, Team, Position (Pos), Age, Matches Played (MP), Starts, Minutes played (Min), Goals scored (Gls\_st), Assists (Ast\_st), Yellow cards (CrdY\_st), Red cards (CrdR\_st), Expected goals (xG\_st), Expected assists (xAG\_st), Progressive carries (PrgC\_st), Progressive passes (PrgP\_st), Progressive passes received (PrgR\_st), Goals per 90 minutes (Gls\_st\_per90), Assists per 90 minutes (Ast\_st\_per90), Expected goals per 90 minutes (xG\_st\_per90), Expected assists per 90 minutes (xAG\_st\_per90).
    \item \textbf{Goalkeeping Stats:} Goals against per 90 minutes (GA90\_gk), Save percentage (\%) (Save\%\_gk), Clean sheet percentage (\%) (CS\%\_gk), Penalty save percentage (\%) (Save\%\_gk\_pen).
    \item \textbf{Shooting Stats:} Shots on target percentage (\%) (SoT\%\_sh), Shots on target per 90 minutes (SoT/90\_sh), Goals per shot (G/Sh\_sh), Average shot distance (Dist\_sh).
    \item \textbf{Passing Stats:} Completed passes (Cmp\_pas), Pass completion rate (\%) (Cmp\%\_pas), Short pass completion rate (\%) (Cmp\%\_pas\_S), Medium pass completion rate (\%) (Cmp\%\_pas\_M), Long pass completion rate (\%) (Cmp\%\_pas\_L), Key passes (KP\_pas), Passes into final third (1/3\_pas), Passes into penalty area (PPA\_pas), Crosses into penalty area (CrsPA\_pas), Progressive passes (PrgP\_pas).
    \item \textbf{Goal and Shot Creation Stats:} Shot-creating actions (SCA\_gsc), Shot-creating actions per 90 minutes (SCA90\_gsc), Goal-creating actions (GCA\_gsc), Goal-creating actions per 90 minutes (GCA90\_gsc).
    \item \textbf{Defensive Actions Stats:} Tackles (Tkl\_def), Tackles won (TklW\_def), Challenges attempted (Att\_def), Challenges lost (Lost\_def), Blocks (Blocks\_def), Shots blocked (Sh\_def), Passes blocked (Pass\_def), Interceptions (Int\_def).
    \item \textbf{Possession Stats:} Touches (Touches\_pos), Touches in defensive penalty area (Def Pen\_pos), Touches in defensive third (Def 3rd\_pos), Touches in middle third (Mid 3rd\_pos), Touches in attacking third (Att 3rd\_pos), Touches in attacking penalty area (Att Pen), Dribbles attempted (Att\_pos), Successful dribble percentage (\%) (Succ\%\_pos), Dispossessed percentage (\%) (Tkld\%\_pos), Ball carries (Carries\_pos), Total progressive carrying distance (PrgDist\_pos), Progressive carries (PrgC\_pos), Carries into final third (1/3\_pos), Carries into penalty area (CPA\_pos), Miscontrols (Mis\_pos), Dispossessed (Dis\_pos), Passes received (Rec\_pos), Progressive passes received (PrgR\_pos).
    \item \textbf{Miscellaneous Stats:} Fouls committed (Fls\_mis), Fouls drawn (Fld\_mis), Offsides (Off\_mis), Crosses (Crs\_mis), Ball recoveries (Recov\_mis), Aerial duels won (Won\_mis), Aerial duels lost (Lost\_mis), Aerial duel win percentage (\%) (Won\%\_mis).
\end{itemize}

The results.csv file contains a list of players with all of the above parameters, helping to analyze or use for deeper reporting on players in the Premier League.

\section{Problem 2:}

\subsection*{A. Problem-2a:}

\textbf{Approach:} \textit{problem-2a} has the task of processing data from the results.csv file (results of problem-1) to find \textit{the top three players and the bottom three players for each statistical indicator} in the data. The implementation process includes the following.

\begin{itemize}
    \item \textit{Reading the data:} The CSV file containing information about Premier League players is loaded into a DataFrame.
    \item \textit{Data processing:}
    
    \begin{itemize}
        \item Remove columns containing non-numeric information (Player, Nation, Team, Pos).
        \item Convert N/a data to 0.0 to ensure calculations do not produce errors.
        \item Convert the values of numeric columns to numeric data types (int64, float64).
    \end{itemize}
    
    \item \textit{Sort and select players:}
    
    \begin{itemize}
        \item Sort each statistical column by value from low to high.
        \item Take the three players with the lowest values and the three players with the highest values for each indicator.
    \end{itemize}
    
    \item \textit{Write results to file:} Save the list of players with top \& bottom 3 for each indicator to the Top\_and\_Bottom\_3\_statistics.txt file.
\end{itemize}

\textbf{Results:} After running the program, the Top\_and\_Bottom\_3\_statistics.txt file will contain a list of \textit{best and worst players} for each statistical indicator in the data, including:

\begin{itemize}
    \item Number of goals, number of assists.
    \item Defensive performance, passing.
    \item Advanced metrics such as xG (expected goals), CS\% (clean sheet percentage for goalkeepers), Tkl\_def (number of tackles), etc.
    \item Special indicators for \textit{passing, saves, dribbling} and many other factors.
\end{itemize}

This file helps easily \textit{analyze strengths \& weaknesses} of each player in the league, supporting statistics, evaluation, or making more in-depth assessments.

\subsection*{B. Problem-2b:}
\textbf{Approach:} \textit{problem-2b} is designed to calculate important statistics for each team in the \textit{Premier League}. It processes data from the results.csv file (results of problem-1) to determine:

\begin{itemize}
    \item \textit{Mean value} -- The average value of each indicator within the team.
    \item \textit{Median value} -- The midpoint of the data, helping to eliminate the influence of outliers.
    \item \textit{Standard Deviation} -- Measuring the variability of each indicator within the team.
\end{itemize}

The key steps include:

\begin{itemize}
    \item \textit{Reading data} from the results.csv file (results of problem-1).
    \item \textit{Removing unnecessary columns}: Player, Nation, Age, Pos, keeping only data by team.
    \item \textit{Processing data}: Converting N/a values to 0.0 and forcing data types to ensure calculations do not encounter errors.
    \item \textit{Calculating statistics} for each team.
    \item \textit{Saving results}: The calculated data are saved to the results.csv file.
\end{itemize}

\textbf{Results:} After running the program, the results.csv file will contain:

\begin{itemize}
    \item List of teams in the \textit{Premier League}.
    \item Mean, median, and standard deviation of each indicator (e.g., goals, assists, shot accuracy percentage, number of tackles, etc.).
    \item An overview of the stability or performance disparity between teams.
\end{itemize}

This file is used to \textit{analyze} and \textit{compare the strength} of teams, as well as identify which teams play consistently and which teams have large performance fluctuations.

\subsection*{C. Problem-2c:}
\textbf{Approach:} \textit{problem-2c} has the task of \textit{creating histograms} to visualize \textit{attacking} and \textit{defensive} indicators of players in the \textit{Premier League}. Specifically, it performs the following steps:

\begin{itemize}
    \item \textit{Reading data} from the results.csv file (results of problem-1) containing player information.
    \item \textit{Data processing:}
    
    \begin{itemize}
        \item Converting relevant data columns (Gls\_st, Ast\_st, xG\_st, Tkl\_def, Int\_def, Blocks\_def) to numeric types.
        \item Filling missing values (NaN) with 0.0 to avoid errors when drawing charts.
    \end{itemize}
    
    \item \textit{Drawing histograms:}
    
    \begin{itemize}
        \item Creating 6 \textit{charts}, divided into 2 \textit{rows}, 3 \textit{columns}.
        \item Each chart represents an indicator:
        
        \begin{itemize}
            \item \textit{Attacking}: Goals (Gls\_st), Assists (Ast\_st), Expected Goals (xG\_st).
            \item \textit{Defensive}: Tackles (Tkl\_def), Interceptions (Int\_def), Blocks (Blocks\_def).
        \end{itemize}
        
        \item Using \textit{different colors} for each chart for easy differentiation.
        \item \textit{Adding titles}, axis labels, and grids to enhance readability.
    \end{itemize}
    
    \item \textit{Saving results}: The chart is saved as histogram\_premier\_league.png for viewing.
\end{itemize}

\textbf{Results:} After running the program, the result is:

\begin{itemize}
    \item \textit{An image containing 6 histograms}, visualizing \textit{Premier League} players' performance by \textit{attacking \& defensive indicators}.
    \item Easy identification of the distribution of \textit{goals, assists, expected goals, tackles, blocks, interceptions} among players.
    \item Support for \textit{performance analysis} to find players with strong attacking play or solid defense.
\end{itemize}

\subsection*{D, Problem-2d:}

\textbf{Approach:} \textit{problem-2d} has the task of \textit{identifying the team with the best performance} in the season based on a set of statistical indicators. It performs the following steps:

\begin{itemize}
    \item \textit{Reading data} from the results.csv file (results of problem-2b), containing average statistics for each team.
    \item \textit{Categorizing indicators}:
    
    \begin{itemize}
        \item \textit{Attack} (\textbf{Atk}): Number of goals (Gls\_st), number of assists (Ast\_st), expected goals (xG\_st).
        \item \textit{Defense} (\textbf{Def}): Number of tackles (Tkl\_def), number of interceptions (Int\_def), number of blocks (Blocks\_def).
        \item \textit{Possession} (\textbf{Pos}): Pass success rate (Cmp\_pas), number of progressive passes (PrgP\_pas), number of touches (Touches\_pos).
    \end{itemize}
    
    \item \textit{Data normalization}: Converting the value of each indicator to the same scale (from 0 to 1) for easier comparison.
    \item \textit{Calculating composite scores}:
    
    \begin{itemize}
        \item \textit{Attack} score accounts for \textbf{40\%} of total score.
        \item \textit{Defense} score accounts for \textbf{30\%} of total score.
        \item \textit{Possession} score accounts for \textbf{30\%} of total score.
    \end{itemize}
    
    \item \textit{Ranking teams} by composite score to determine \textit{the team with the best performance in the season}.
    \item \textit{Saving results} to the Team\_stat\_leaders\_and\_record.txt file.
\end{itemize}

\textbf{Results:} After running the program, the results include:

\begin{itemize}
    \item \textit{List of teams} ranked by composite score from highest to lowest.
    \item An overview of the \textit{best performing teams}, based on \textit{attacking performance, defense, and ball possession}.
    \item The team with the \textit{best performance} in the season, clearly displayed with their composite score.
\end{itemize}

This file helps \textit{compare strength between teams}, find the team with the strongest attacking style, the most solid defense, or the team with the best match control.

\section{Problem 3:}

\subsection*{A, Problem-3a:}

\textbf{Approach:} \textit{problem-3a} uses the \textit{Elbow Method} to find the optimal number of clusters (K) in \textit{K-Means clustering}. Its main objectives are:

\begin{itemize}
    \item \textit{Reading data} from the results.csv file (results of problem-1), which contains statistical information about players.
    \item \textit{Data preprocessing}:
    
    \begin{itemize}
        \item Selecting columns containing numerical values.
        \item Normalizing data using \textit{StandardScaler} to ensure features have the same scale, helping the clustering algorithm work better.
    \end{itemize}
    
    \item \textit{K-Means clustering:}
    
    \begin{itemize}
        \item Running K-Means with number of clusters from \textit{1 to 10}.
        \item Recording \textit{intra-group variability} (Inertia) for each cluster number K.
    \end{itemize}
    
    \item \textit{Drawing the Elbow chart:}
    
    \begin{itemize}
        \item X-axis: Number of clusters (K).
        \item Y-axis: Value of \textit{intra-group variability} (Inertia).
        \item Finding the Elbow point, where variability begins to decrease more slowly (this is the optimal K value).
    \end{itemize}
    
    \item \textit{Saving} the chart as Find\_the\_optimal\_k.png.
\end{itemize}

\textbf{Results:} After running the program, the results include:

\begin{itemize}
    \item \textit{An Elbow chart}, helping to determine the most suitable number of clusters in player data analysis.
    \item Results supporting the selection of \textit{optimal K}, helping classify players based on statistical characteristics.
    \item A scientific approach to grouping players with similar playing styles.
\end{itemize}

\textbf{Comments on results:} Based on the \textit{Elbow Method} chart from Find\_the\_optimal\_k.png, the \textit{Elbow} point appears at K = 3, meaning this method suggests that 3 \textit{groups} is the optimal number of clusters to classify players. This is the level where intra-group variability (inertia) decreases sharply, but then the rate of decrease slows down as the number of clusters increases. The significance of 3 groups may be \textit{outstanding attacking players} with notable goal-scoring and assist abilities, \textit{strong defensive players} possessing good tackling, interception, and blocking skills, and \textit{good ball control players} with high pass rates and effective ball retention.

\subsection*{B, Problem-3b:}

\textbf{Approach:} \textit{problem-3b} uses \textit{K-Means} clustering combined with \textit{Principal Component Analysis} (PCA) to classify players based on their statistics. The main objectives are:

\begin{itemize}
    \item \textit{Data Processing:}
    
    \begin{itemize}
        \item Reading data from results.csv (output from problem-1) and filtering columns containing player statistics.
        \item Normalizing data using StandardScaler to ensure equal scale for all features.
        \item Using PCA to reduce dimensionality to 2 \textit{principal components}, making visualization easier.
    \end{itemize}
    
    \item \textit{Clustering with K-Means:}
    
    \begin{itemize}
        \item Identifying 3 \textit{player groups} (according to results from the \textit{Elbow} method).
        \item Assigning each player to a cluster based on their statistical characteristics.
    \end{itemize}
    
    \item \textit{Drawing the clustering chart:}
    
    \begin{itemize}
        \item Displaying player data on the axis system \textit{PCA Component 1 \& Component 2.}
        \item Using colors (viridis color palette) to distinguish groups.
        \item Marking \textit{centroids} in red to see the average position of each group.
    \end{itemize}
    
    \item \textit{Saving} the chart as KMeans\_clustering.png.
    
\end{itemize}
\textbf{Results:} The program outputs include:

\begin{itemize}
    \item \textit{A 2D chart} showing player clustering based on their statistical data.
    \item \textit{Three clear groups}, which may represent:
    
    \begin{itemize}
        \item Group  1: Attacking players (good at shooting, assisting).
        \item Group  2: Defensive players (good at tackling, blocking).
        \item Group  3: Ball control players (good at passing and ball retention).
    \end{itemize}
    
    \item Easy identification of \textit{common characteristics} of each group, helping analyze and optimize tactics.
\end{itemize}

\section{Problem 4}

\subsection*{A, Problem-4a}

\textbf{Approach:} \textit{problem-4a} performs \textit{data collection on transfer values of Premier League} players from the website \href{https://www.footballtransfers.com/en}{FootballTransfers.com} using \textit{web scraping} with Selenium. The main objectives are:

\begin{itemize}
    \item \textit{Reading data} from the results.csv file (results of problem-1), getting a list of players including: Player, Pos, Team, Age, and Min > 900.
    \item \textit{Setting up Selenium browser}:
    
    \begin{itemize}
        \item Configuring \textit{Chrome} browser to run \textit{headless} (no window display).
        \item \textit{Automatically navigating} to the FootballTransfers website.
        \item \textit{Disabling notifications}, avoiding SSL certificate errors.
    \end{itemize}
    
    \item \textit{Interacting with the website:}
    
    \begin{itemize}
        \item \textit{Closing notification pop-ups} if they appear.
        \item \textit{Moving the mouse to} "Players" section.
        \item \textit{Clicking on} "All Premier League Players" to access the player list page.
    \end{itemize}
    
    \item \textit{Collecting data:}
    
    \begin{itemize}
        \item Browsing through each player list page.
        \item \textit{Getting information} about Skill, Pot, and transfer value (ETV).
        \item Saving data to the all\_players list.
    \end{itemize}
    
    \item \textit{Saving results} to results.csv.
\end{itemize}

\textbf{Results:} After running the program, the results include:

\begin{itemize}
    \item \textit{A list of Premier League players} with information:
    
    \begin{itemize}
        \item \textit{Player, Pos, Team, Age, Min}.
        \item \textit{Skill} and \textit{Pot}, showing level of expertise.
        \item \textit{Expected Transfer Value} (ETV) from FootballTransfers.
    \end{itemize}
    
    \item \textit{CSV file containing players meeting criteria} (>900 minutes played), useful for data analysis or machine learning models.
\end{itemize}

\subsection*{B, Problem-4b}

\textbf{Approach:} \textit{problem-4b} helps \textit{build a player valuation model} based on statistical parameters. It combines data on \textit{Min, Age, Skill, Pot} with actual transfer values (ETV) to predict player value.
Key steps:

\begin{itemize}
    \item \textit{Reading data} from results.csv (results of problem-1) containing player statistics and results.csv (results of problem-4a) containing transfer values.
    \item \textit{Data processing}:
    
    \begin{itemize}
        \item \textit{Normalizing ETV} by extracting real numbers from strings.
        \item \textit{Converting Age} to numeric form (years + days/365) to increase accuracy.
        \item \textit{Separating skill and potential information} (Skill / Pot).
        \item \textit{One-hot encoding for Pos} to turn player positions into numeric features.
    \end{itemize}
    
    \item \textit{Building machine learning model:}
    
    \begin{itemize}
        \item Choosing \textit{Linear Regression} to predict player value based on input factors.
        \item Applying \textit{log transformation} (log1p(ETV)) to handle skewed distribution of transfer values.
        \item Splitting the dataset into \textit{train} (80\%) and \textit{test} (20\%).
    \end{itemize}
    
    \item \textit{Evaluating the model:}
    
    \begin{itemize}
        \item Calculating \textit{RMSE} (Root Mean Squared Error) to measure the difference between actual and predicted values.
        \item Calculating \textit{R² Score}, reflecting the model's goodness of fit.
    \end{itemize}
    
    \item \textit{Exporting results}: Comparing predicted values with actual values, saving to predicted\_vs\_actual.csv.
\end{itemize}

\textbf{Results}: The results obtained after running the model will include:

\begin{itemize}
    \item \textit{Predicted \& actual values} of player transfer values in predicted\_vs\_actual.csv.
    \item \textit{RMSE} $\approx$ 0.3954 (low) shows \textit{model has small error}, meaning player transfer value predictions are relatively accurate.
    \item \textit{R² Score} $\approx$ 0.7786 (high) shows the model reflects the player valuation trend well.
\end{itemize}

\textbf{Feature and Model Selection:}

\begin{itemize}
    \item Feature selection: Features are selected based on \textit{potential relationship with player value}:
    
    \begin{itemize}
        \item \textit{Age}: Younger players typically have higher value.
        \item \textit{Min}: Players who play more often are typically valued higher.
        \item \textit{Skill} and \textit{Pot}: Higher scores indicate capability.
        \item \textit{Pos}: An excellent striker typically has higher value than a defender.
    \end{itemize}
    
    \item Model selection: \textit{Linear Regression} was chosen because:
    
    \begin{itemize}
        \item Player value often has a linear relationship with performance indicators.
        \item Easy to interpret and suitable for small-scale data.
        \item Can be improved by applying \textit{log transformation} to handle skewed distribution.
    \end{itemize}
    
\end{itemize}

\end{document}